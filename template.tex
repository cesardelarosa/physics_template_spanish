\documentclass[twocolumn, aps, prd, 10pt, nofootinbib]{revtex4-2}
\usepackage[T1]{fontenc}
\usepackage[utf8]{inputenc}
%\usepackage[protrusion=true, expansion=true]{microtype}
\usepackage[spanish,es-noshorthands]{babel}
\let\ls\lstinline
\usepackage{
	amsmath, amssymb, graphicx, float, hyperref, booktabs, geometry, siunitx,
	pgfplots, pgfplotstable, cleveref, listings, setspace, xcolor, tikz
}

\pgfplotsset{compat=1.18}
\geometry{margin=1.7cm}

\lstset{ 
    language=Matlab,								% El lenguaje de programación a mostrar
    basicstyle=\ttfamily\small\color{black!90},		% Fuente principal
    keywordstyle=\color{cyan!70!black}\bfseries,	% Color y estilo de palabras clave
    commentstyle=\color{green!50!black}\itshape,	% Color y estilo de comentarios
    stringstyle=\color{magenta!90!black},			% Color de strings
    numbers=left,									% Números de línea a la izquierda
    numberstyle=\tiny\color{gray},					% Estilo de los números de línea
    frame=lines,									% Cuadro superior e inferior
    rulecolor=\color{black!20},						% Color de los bordes
    backgroundcolor=\color{black!5},				% Fondo del código
    breaklines=true,								% Romper líneas largas
    breakatwhitespace=false,						% Romper en espacios en blanco
    showstringspaces=false,							% No mostrar espacios en strings
    tabsize=4,										% Tamaño de tabulación
    literate={á}{{\'a}}1 {é}{{\'e}}1 {í}{{\'i}}1 {ó}{{\'o}}1 {ú}{{\'u}}1
             {Á}{{\'A}}1 {É}{{\'E}}1 {Í}{{\'I}}1 {Ó}{{\'O}}1 {Ú}{{\'U}}1
             {ñ}{{\~n}}1 {Ñ}{{\~N}}1 {¿}{{\textquestiondown}}1 {¡}{{\textexclamdown}}1
             {_}{{\_}}1,							% Manejo del castellano
    captionpos=b									% Posición del título del código (abajo)
}

\makeatletter
\renewcommand{\date}[1]{\gdef\@date{\textnormal{Fecha: #1}}}
\date{\today}
\makeatother

\begin{document}

\title{Título}
\author{César de la Rosa Sobrino}
\email{cdelarosa29@alumno.uned.es}
\affiliation{Univerdad Nacional de Educación a Distancia (UNED)\\Física Cuántica I}
\date{\today}

\begin{abstract}
	Resumen
\end{abstract}

\maketitle

\section{Introducción}
Introducción.
\section{Fundamento teórico}
Fundamento teórico.

\begin{equation}
	E = mc^2
\end{equation}
\cite{einstein1905}

\begin{align}
  x + y &= 10, \\
  2x - y &= 3.
\end{align}

\begin{equation}
  \begin{bmatrix}
    a & b \\
    c & d
  \end{bmatrix}
  \begin{bmatrix}
    x \\ y
  \end{bmatrix}
  =
  \begin{bmatrix}
    0 \\ 1
  \end{bmatrix}
\end{equation}

\begin{equation}
  \sum_{n=1}^{\infty} \frac{1}{n^2} = \frac{\pi^2}{6}
\end{equation}

\begin{equation}
  \frac{d^2y}{dx^2} + 3 \frac{dy}{dx} + 2y = 0
\end{equation}

\section{Resultados}
Resultados.


\begin{table}[H]
  \centering
  \caption{Datos desde un archivo CSV}
  \pgfplotstabletypeset[
    col sep=comma,
    display columns/0/.style={column name={Variable X}},
    display columns/1/.style={column name={Variable Y}},
    every head row/.style={before row=\toprule, after row=\midrule},
    every last row/.style={after row=\bottomrule}
  ]{datos.csv}
  \label{tabla:csv}
\end{table}

\begin{figure}[H]
  \centering
  \caption{Gráfico de ejemplo a partir de un archivo de datos}
  \begin{tikzpicture}
    \begin{axis}[
      xlabel={Eje X},
      ylabel={Eje Y},
      grid=major,
      width=0.45\textwidth
    ]
      \addplot table [col sep=comma, x index=0, y index=1] {datos.csv};
    \end{axis}
  \end{tikzpicture}
  \label{fig:ejemplo}
\end{figure}

\section{Conclusiones}
Conclusiones.

\bibliographystyle{apsrev4-2}
\bibliography{referencias}

\cleardoublepage
\appendix
\section{Código}

\begin{figure}[H]
\centering
\lstinputlisting[caption={sample.m - Cálculo de la raíz cuadrada de un vector en MATLAB}, label={lst:sample}]{sample.m}
\end{figure}

\end{document}
